\documentclass[12pt]{article}
\usepackage[margin=0.6in]{geometry}
\usepackage{amsmath}
\usepackage{xcolor}
\usepackage{verbatim}
\usepackage{listings}
\usepackage{graphicx,caption}
\usepackage{ragged2e}
\usepackage{MnSymbol}
\usepackage{wasysym}
\usepackage{hyperref}
\graphicspath{{./images/}}

\begin{document}

\section*{Major Project Title}

Federated Learning with Verifiable Clients and Aggregators

\section*{Proposed Cluster}
\begin{itemize}
    \item AIML
    \item BD
\end{itemize}

\section*{Team Members}
\begin{itemize}
    \item Bindu Paudel (078BCT032)
    \item Krishant Timilsina (078BCT045)
\end{itemize}

\section*{Project Overview}
\justifying
This project proposes a novel Federated Learning (FL) framework based on the paper \textit{zkFL: Zero-Knowledge Federated Learning with Verifiable Clients and Aggregators} \cite{xu2021zkfl}. The core idea is to enhance the trustworthiness of federated learning through cryptographic enforcement using Zero-Knowledge Proofs (ZKPs).

In zkFL, each client locally trains its model and generates a succinct ZKP that the training was done correctly on genuine data. The aggregator, rather than just blindly summing model updates, performs a verifiable aggregation process and produces its own ZKP ensuring that it only aggregated valid updates from honest clients.

Our implementation will simulate clients using the Flower or TensorFlow Federated framework and incorporate zk-SNARKs or Halo2 for the proof system. Clients will submit their model updates along with proofs, while the server verifies and aggregates them, also generating a global ZKP of correct aggregation. This dual-verification approach strengthens FL by preventing both client- and server-side tampering, ensuring privacy, integrity, and auditability—making it suitable for sensitive domains such as healthcare (e.g., using the MIMIC-III dataset) and autonomous vehicle collaboration.

\section*{Motivation}
\justifying
Current Federated Learning systems either assume the aggregator is honest or only verify client-side updates. However, adversaries can exist on both ends. Combining ZKP-based mechanisms for both clients and servers allows us to develop a system where trust is cryptographically enforced, enabling secure deployments in critical domains without sacrificing privacy or performance \cite{li2020federated}.

\section*{Implementation Plan}
\justifying
We will simulate multiple clients using a federated learning framework (e.g., Flower or TFF), integrating ZKPs for both local training proofs and secure aggregation. The implementation will be benchmarked using real datasets such as MIMIC-III (ICU data) \cite{johnson2016mimic}. \newline

\includegraphics[width=\linewidth]{zkfl_architecture.png} \\
\captionof{figure}{Proposed architecture: ZKPs from clients prove honest training; server generates ZKP of correct aggregation.}

\section*{Use Cases}
\begin{itemize}
    \item Secure healthcare prediction using ICU records (e.g., mortality prediction)
    \item Collaborative autonomous driving models without raw data sharing
    \item Privacy-preserving smart device learning in IoT environments
\end{itemize}

\section*{Resource Required}
\begin{itemize}
    \item High-performance computing system with 16 GB RAM, GPU (NVIDIA GTX 1660 or higher)
    \item Python-based libraries: PyTorch/TensorFlow, Flower/TFF, zkSNARK libraries (e.g., libsnark, Halo2)
    \item Public datasets (e.g., MIMIC-III, CIFAR-10)
    \item Optional: Docker, Git, VS Code, Colab Pro for collaborative development
\end{itemize}

\bibliographystyle{IEEEtran}
\bibliography{references}

\end{document}
