\chapter{Progress Report}

This chapter provides a summary of the project's current status, completed work, and future directions for the secure federated learning framework.

\section{Project Status Summary}

The Secure Federated Learning with Zero-Knowledge Proofs project has achieved functional implementation of a dual-verifiable federated learning system with demonstrated security properties and comprehensive experimental validation.

\section{Completed Work}

\subsection{Core Implementation}

\textbf{Secure Federated Learning Framework:} Complete implementation of dual-verification system with:
\begin{itemize}
    \item Client-side zero-knowledge proofs using PySNARK
    \item Server-side aggregation proofs using Groth16
    \item FedJSCM momentum-based aggregation algorithm
    \item Complete end-to-end cryptographic verification pipeline
\end{itemize}

\textbf{Production Components:} Functional system components including:
\begin{itemize}
    \item SecureFlowerClient and SecureFlowerServer implementations
    \item ClientProofManager and ServerProofManager modules
    \item Comprehensive testing and validation framework
    \item Performance benchmarking and analysis tools
\end{itemize}

\textbf{Experimental Validation:} Complete testing across multiple datasets demonstrating system functionality and performance characteristics, with measured 759x proof generation overhead.

\subsection{Current Status}

The project has successfully completed core development milestones:

\textbf{System Implementation:} All core components are functional and tested
\begin{itemize}
    \item Dual zero-knowledge proof architecture working end-to-end
    \item FedJSCM aggregation algorithm implemented and validated
    \item Multi-dataset testing completed across 8 diverse datasets
    \item Comprehensive performance characterization completed
\end{itemize}

\textbf{Performance Validation:} System demonstrated with quantified overhead
\begin{itemize}
    \item Proof overhead measured at 759x training time
    \item 8 different datasets tested successfully
    \item Security properties validated through cryptographic verification
    \item Deployment feasibility assessed for high-value applications
\end{itemize}

\section{Current Work}

\subsection{Ongoing Activities}

\textbf{Performance Optimization:} Continuing to improve system efficiency through circuit optimization and caching mechanisms to reduce the significant proof generation overhead.

\textbf{Documentation:} Completing comprehensive technical documentation and deployment guides for broader adoption.

\textbf{Testing Enhancement:} Expanding test coverage and validation across additional scenarios and datasets.

\section{Future Work and Limitations}

\textbf{Future Work (Priority Items):}
\begin{itemize}
    \item \textbf{Dynamic Proof Adjustment}: Implementation of adaptive rigor levels (high/medium/low) based on training stability to reduce computational overhead
    \item Circuit optimization and hardware acceleration for improved performance
    \item Advanced proof caching and reuse mechanisms
    \item Integration with differential privacy mechanisms
    \item Cross-platform optimization for mobile and edge environments
    \item Advanced threat detection and response systems
    \item Integration with additional ML frameworks
\end{itemize}

\textbf{Current Limitations:}
\begin{itemize}
    \item High proof generation overhead (759x) limits practical deployment
    \item Memory requirements may limit resource-constrained deployments
    \item Focus primarily on supervised learning scenarios
    \item Requires stable network connectivity for verification
    \item Setup complexity requires technical expertise
\end{itemize}

\section{Summary}

The project has successfully developed and validated a dual-verification federated learning framework combining client-side and server-side zero-knowledge proofs. While the system demonstrates strong cryptographic security guarantees and functionality across multiple datasets, the significant computational overhead (759x) represents a key limitation that requires future optimization work. The dynamic proof adjustment system remains a priority for implementation to achieve practical deployment viability.
